\documentclass[a4paper]{article}\usepackage[]{graphicx}\usepackage[]{xcolor}
% maxwidth is the original width if it is less than linewidth
% otherwise use linewidth (to make sure the graphics do not exceed the margin)
\makeatletter
\def\maxwidth{ %
  \ifdim\Gin@nat@width>\linewidth
    \linewidth
  \else
    \Gin@nat@width
  \fi
}
\makeatother

\definecolor{fgcolor}{rgb}{0.345, 0.345, 0.345}
\newcommand{\hlnum}[1]{\textcolor[rgb]{0.686,0.059,0.569}{#1}}%
\newcommand{\hlstr}[1]{\textcolor[rgb]{0.192,0.494,0.8}{#1}}%
\newcommand{\hlcom}[1]{\textcolor[rgb]{0.678,0.584,0.686}{\textit{#1}}}%
\newcommand{\hlopt}[1]{\textcolor[rgb]{0,0,0}{#1}}%
\newcommand{\hlstd}[1]{\textcolor[rgb]{0.345,0.345,0.345}{#1}}%
\newcommand{\hlkwa}[1]{\textcolor[rgb]{0.161,0.373,0.58}{\textbf{#1}}}%
\newcommand{\hlkwb}[1]{\textcolor[rgb]{0.69,0.353,0.396}{#1}}%
\newcommand{\hlkwc}[1]{\textcolor[rgb]{0.333,0.667,0.333}{#1}}%
\newcommand{\hlkwd}[1]{\textcolor[rgb]{0.737,0.353,0.396}{\textbf{#1}}}%
\let\hlipl\hlkwb

\usepackage{framed}
\makeatletter
\newenvironment{kframe}{%
 \def\at@end@of@kframe{}%
 \ifinner\ifhmode%
  \def\at@end@of@kframe{\end{minipage}}%
  \begin{minipage}{\columnwidth}%
 \fi\fi%
 \def\FrameCommand##1{\hskip\@totalleftmargin \hskip-\fboxsep
 \colorbox{shadecolor}{##1}\hskip-\fboxsep
     % There is no \\@totalrightmargin, so:
     \hskip-\linewidth \hskip-\@totalleftmargin \hskip\columnwidth}%
 \MakeFramed {\advance\hsize-\width
   \@totalleftmargin\z@ \linewidth\hsize
   \@setminipage}}%
 {\par\unskip\endMakeFramed%
 \at@end@of@kframe}
\makeatother

\definecolor{shadecolor}{rgb}{.97, .97, .97}
\definecolor{messagecolor}{rgb}{0, 0, 0}
\definecolor{warningcolor}{rgb}{1, 0, 1}
\definecolor{errorcolor}{rgb}{1, 0, 0}
\newenvironment{knitrout}{}{} % an empty environment to be redefined in TeX

\usepackage{alltt}

% % \usepackage{fontspec}
% \usepackage{polyglossia}
\usepackage[french]{babel}
\usepackage{fancyhdr}
\usepackage{geometry}

\usepackage{amsmath}
\usepackage{amssymb}

\usepackage{xcolor}

\geometry{a4paper,left=15mm,right=15mm,top=20mm,bottom=20mm}
\pagestyle{fancy}
\chead{Feuille d'exercices systèmes et matrices}
\rhead{\today}
% \cfoot{\thepage}

\setlength{\headheight}{23pt}
\setlength{\parindent}{0.0in}
\setlength{\parskip}{0.0in}

\newcommand{\R}{\mathbb{R}}
\IfFileExists{upquote.sty}{\usepackage{upquote}}{}
\begin{document}



\section*{Système}

\vspace{0,75cm}

\subsection*{Systèmes compatibles}



\begin{equation}
    \begin{cases} x_1 + 4x_2 + 2x_3 + 3x_4  =  1 \\ 3x_1 + 2x_3 + 2x_4  =  2 \\ 2x_1 + 2x_2 + 5x_3  =  3 \\ 4x_1 + 4x_2 + 2x_3 + 4x_4  =  4 \\ \end{cases}
    \label{eqn:1e-systeme}
\end{equation}

Le système~\ref{eqn:2e-systeme} décrit les réactions chimiques entre 3 composés $x_1, x_2, x_3$ et le
second membre des équations donne les quantités d'un 4e composant pour lesquels 
un patient est considéré en bonne santé. 

\noindent\begin{minipage}{.5\linewidth}


\begin{equation}
    \begin{cases} x_1 + 5x_2 + 2x_3  =  4.3 \\ 4x_2 + 6x_3  =  6.8 \\ x_1 + 3x_2 + 2x_3  =  3.6 \\ \end{cases}
    \label{eqn:2e-systeme}
\end{equation}
\end{minipage}
\begin{minipage}{.5\linewidth}


\begin{equation}
    \begin{cases} 2x_1 + x_2 + 5x_3  =  6.35 \\ x_3  =  0.9 \\ 7x_1 + x_2  =  5.6 \\ \end{cases}
    \label{eqn:3e-systeme}
\end{equation}
\end{minipage}

Résoudre les systèmes \ref{eqn:1e-systeme}, \ref{eqn:2e-systeme}, \ref{eqn:3e-systeme} et donner les valeurs des inconnues.

\subsection*{Systèmes avec inconnues secondaires}
\begin{minipage}{0.33\linewidth}
    



\begin{equation}
    \begin{cases} x_1 + 3x_2  =  5 \\ 2x_2 + x_3  =  2 \\ \end{cases}
    \label{eqn:4e-systeme}
\end{equation}
\end{minipage}
\begin{minipage}{0.33\linewidth}


\begin{equation}
    \begin{cases} 2x_1 + x_2 + x_3  =  0 \\ 5x_1 + 2x_2 + 3x_3 + 5x_4  =  2 \\ \end{cases}
    \label{eqn:5e-systeme}
\end{equation}
\end{minipage}
\begin{minipage}{0.33\linewidth}


\begin{equation}
    \begin{cases} 4x_1 + 3x_2 + x_3  =  1 \\ 4x_1  =  2 \\ \end{cases}
    \label{eqn:6e-systeme}
\end{equation}
\end{minipage}

Résoudre les systèmes \ref{eqn:4e-systeme}, \ref{eqn:5e-systeme}, \ref{eqn:6e-systeme} puis :
\begin{enumerate}
    \item Exprimer les ensembles de solutions sous la forme $S = \{(x, y, z) , z \in \R\}$ 
    (En remplaçant les inconnues par la fonction de l'inconnue secondaire correspondante)
    \item Exprimer maintenant les ensembles de solutions sous la forme d'équation de droites ou de plans.\\
    $S = \{ \vec{a} + \vec{b} z \}$ par exemple, où $\vec{a}$ et $\vec{b}$ sont des vecteurs (c'est-à-dire quelque chose de la forme $(1;2;3)$).
    \item Dire si le système est une équation de droite ou bien de plan en justifiant.
\end{enumerate}

Exemple résolu pour le système \ref{eqn:4e-systeme} :
\begin{enumerate}
    \item $S_4 = \{(2 + \frac{3}{2}z, 1 - \frac{1}{2} z, z), z\in\R\}$
    \item $S_4^* = \{(2, 1, 0) + (\frac{3}{2},  - \frac{1}{2} , 1)z, z\in\R\}$

\end{enumerate}

\section*{Matrices}

\subsection*{Pour s'échauffer}



\begin{minipage}{0.33\linewidth}
    \begin{equation*}
        A = \begin{bmatrix} 4 &6 \\4 &5 \\ \end{bmatrix}
    \end{equation*}
\end{minipage}
\begin{minipage}{0.33\linewidth}
    \begin{equation*}
        B = \begin{bmatrix} 8 &1 &4 \\7 &1 &0 \\ \end{bmatrix}
    \end{equation*}
\end{minipage}
\begin{minipage}{0.33\linewidth}
    \begin{equation*}
        C = \begin{bmatrix} 8 &8 &5 \\4 &8 &9 \\3 &5 &7 \\ \end{bmatrix}
    \end{equation*}
\end{minipage}

\begin{enumerate}
    \item Quels sont les produits matriciels possibles en combinant une seule 
    fois $A, B, C$ ? 
    \item Peut on faire plus de produits si on prend la 
    transposée des matrices ? Si oui, calculer les nouveaux produits.
    \item Faire ces produits.
\end{enumerate}

\subsection*{Matrices particulières}



Voici plusieurs matrices :

\begin{minipage}{0.25\linewidth}
    \begin{equation*}
        D = \begin{bmatrix} 5 &5 &6 \\0 &5 &1 \\ \end{bmatrix}
    \end{equation*}
\end{minipage}
\begin{minipage}{0.25\linewidth}
    \begin{equation*}
        D^T = \begin{bmatrix} 5 &0 \\5 &5 \\6 &1 \\ \end{bmatrix}
    \end{equation*}
\end{minipage}
\begin{minipage}{0.25\linewidth}
    \begin{equation*}
        E = \begin{bmatrix} 86 &31 \\31 &26 \\ \end{bmatrix}
    \end{equation*}
\end{minipage}
\begin{minipage}{0.25\linewidth}
    \begin{equation*}
        F = \begin{bmatrix} 25 &25 &30 \\25 &50 &35 \\30 &35 &37 \\ \end{bmatrix}
    \end{equation*}
\end{minipage}

\begin{enumerate}
    \item Faire le produit $DD^T$ et $D^TD$. Que remarque-t-on ?
    \item Les matrices $E$ et $F$ ont une structure particulière, comment 
    appelle-t-on ce type de matrice ?
\end{enumerate}


\newpage

\section*{Correction de la section Système}
Si après avoir re-vérifié tes calculs tu ne trouves pas le même résultat que moi
il est possible que je me sois trompé. Contacte-moi si besoin.

\vspace{0,75cm}

\subsection*{Systèmes compatibles}

\textcolor{blue}{Pour résoudre les systèmes on applique le pivot et on obtient :
\begin{itemize}
    \item Pour le système \ref{eqn:1e-systeme} : $(1.25, 0.5625, -0.125, -0.75)$
    \item Pour le système \ref{eqn:2e-systeme} : $(0.75, 0.35, 0.9)$
    \item Pour le système \ref{eqn:3e-systeme} : $(0.75, 0.35, 0.9)$
\end{itemize}}

\subsection*{Systèmes avec inconnues secondaires}
\textcolor{blue}{Le système \ref{eqn:4e-systeme} est résolu simplement et les écritures des 
ensembles de solutions sont fournies.}\\

\textcolor{blue}{Le système \ref{eqn:5e-systeme} :
\begin{enumerate}
    \item $S_5 = \{(2 - x_3 - 5 x_4, -4 + x_3 + 10 x_4, x_3, x_4), x_3\in\R, x_4 \in \R \}$
    \item $S_5^* = \{(2, -4, 0, 0) + (-1, 1, 1, 0) x_3 + (-5, 10, 0, 1) x_4, x_3\in\R, x_4 \in \R \}$
    \item Il y a 2 inconnues secondaires $x_3$ et $x_4$, il s'agit donc d'une équation de plan dans $\R^4$.
\end{enumerate}}

\textcolor{blue}{Le système \ref{eqn:6e-systeme} :
\begin{enumerate}
    \item $S_6 = \{(\frac{1}{2}, -\frac{1}{3} -\frac{1}{3} x_3, x_3), x_3\in\R\}$
    \item $S_6^* = \{(\frac{1}{2}, - \frac{1}{3}, 0) + (0, -\frac{1}{3}, 1) x_3, x_3\in\R\}$
    \item Il y a 1 inconnue secondaire $x_3$, il s'agit donc d'une équation de droite dans $\R^3$ (le volume, c'est-à-dire à 3 dimensions $(x,y,z)$).
\end{enumerate}}

\section*{Matrices}

\subsection*{Pour s'échauffer}

\begin{minipage}{0.33\linewidth}
    \begin{equation*}
        A = \begin{bmatrix} 4 &6 \\4 &5 \\ \end{bmatrix}
    \end{equation*}
\end{minipage}
\begin{minipage}{0.33\linewidth}
    \begin{equation*}
        B = \begin{bmatrix} 8 &1 &4 \\7 &1 &0 \\ \end{bmatrix}
    \end{equation*}
\end{minipage}
\begin{minipage}{0.33\linewidth}
    \begin{equation*}
        C = \begin{bmatrix} 8 &8 &5 \\4 &8 &9 \\3 &5 &7 \\ \end{bmatrix}
    \end{equation*}
\end{minipage}

\begin{enumerate}
    \item Quels sont les produits matriciels possibles en combinant une seule 
    fois $A, B, C$ ? 
    \item Peut on faire plus de produits si on prend la 
    transposée des matrices ? Si oui, calculer les nouveaux produits.
    \item Faire ces produits.
\end{enumerate}

\textcolor{blue}{Pour savoir quels produits sont possibles il faut regarder si le 
nombre de colonnes de la première matrice correspond au nombre de lignes 
de la seconde. \\
$AB$ est possible, $A$ a 2 colonnes et $B$ 2 lignes. Mais $BA$ lui n'est 
pas possible !
En prenant la transposée de $B$, $B^T = \begin{bmatrix} 8 &7 \\1 &1 \\4 &0 \\ \end{bmatrix}$, on peut alors faire 
le calcul $B^TA$.
\begin{align*}
    AB &= \begin{bmatrix} 74 &10 &16 \\67 &9 &16 \\ \end{bmatrix} & B^TA = \begin{bmatrix} 60 &83 \\8 &11 \\16 &24 \\ \end{bmatrix} \text{ } &  A^TB = \begin{bmatrix} 60 &8 &16 \\83 &11 &24 \\ \end{bmatrix} & B^T A^T = \begin{bmatrix} 74 &67 \\10 &9 \\16 &16 \\ \end{bmatrix}  \\ 
    BC &= \begin{bmatrix} 80 &92 &77 \\60 &64 &44 \\ \end{bmatrix} & CB^T = \begin{bmatrix} 92 &64 \\76 &36 \\57 &26 \\ \end{bmatrix} \text{ } & C^TB^T = \begin{bmatrix} 80 &60 \\92 &64 \\77 &44 \\ \end{bmatrix}\\
\end{align*}}

\subsection*{Matrices particulières}

Voici plusieurs matrices :

\begin{minipage}{0.25\linewidth}
    \begin{equation*}
        D = \begin{bmatrix} 5 &5 &6 \\0 &5 &1 \\ \end{bmatrix}
    \end{equation*}
\end{minipage}
\begin{minipage}{0.25\linewidth}
    \begin{equation*}
        D^T = \begin{bmatrix} 5 &0 \\5 &5 \\6 &1 \\ \end{bmatrix}
    \end{equation*}
\end{minipage}
\begin{minipage}{0.25\linewidth}
    \begin{equation*}
        E = \begin{bmatrix} 86 &31 \\31 &26 \\ \end{bmatrix}
    \end{equation*}
\end{minipage}
\begin{minipage}{0.25\linewidth}
    \begin{equation*}
        F = \begin{bmatrix} 25 &25 &30 \\25 &50 &35 \\30 &35 &37 \\ \end{bmatrix}
    \end{equation*}
\end{minipage}

\begin{enumerate}
    \item Faire le produit $DD^T$ et $D^TD$. Que remarque-t-on ?
    \textcolor{blue}{On remarque que $DD^T = E$ et $D^TD = F$.}
    \item Les matrices $E$ et $F$ ont une structure particulière, comment 
    appelle-t-on ce type de matrice ?
    \textcolor{blue}{Ces matrices sont symétriques, c'est-à-dire égales à leur 
    transposée.
    $$E^T = \begin{bmatrix} 86 &31 \\31 &26 \\ \end{bmatrix} = E$$
    $$F^T = \begin{bmatrix} 25 &25 &30 \\25 &50 &35 \\30 &35 &37 \\ \end{bmatrix} = F$$ 
    On peut décomposer le calcul comme ceci grâce aux propriétés de la 
    transposée:
    $$E = DD^T\text{ et }E^T = (DD^T)^T = (D^T)^T D^T = DD^T$$
    On peut faire le même calcul pour $F$. N'hésite pas à le faire pour t'en 
    convaincre.}
\end{enumerate}

\end{document}
